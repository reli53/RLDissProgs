             % 28.11.88 (Liermann)
% Bemerkungen:
      % Ansehen:  texnet : vu startrl 180
      %                    vu startrl 300   (default)
      %           texram : vu startrl 180   (default)
      %                    vu startrl 300
      % Drucken: Matrix-Drucker NEC p6 : prt startrl (Font 4 einstellen)
      %          HP-Laser-Jet          : hprt startrl
      %          Der andere Druckertreiber (auf C:\LJTEX Terminal Nr.6)
      %                        1.)   dvihp startrl (erzeugt startrl.hp)
      %                        2.)   endspool
      %                        3.)   copy startrl.hp lpt1 /b
      %                        4.)   spool nb nff  (????)
      % Break:   1.) F6   und 2.) F7 druecken!

\documentstyle[12pt]{report}

   % Aufruf der deutschen Ueberschriften + der deutschen Trennung
   \german

%  % Neue Definitionen in LATEX
%            % Seitenformate  ( 1 inch = 2.54 cm )
%  \setlength{\textwidth}{14.9cm}
%  \setlength{\oddsidemargin}{0.4cm}  % 0cm = 1 inch vom linken Rand
%  \setlength{\textheight}{21.6cm}
%  \setlength{\topmargin}{+0.5cm}   % 0cm = 1.75 inch vom oberen Rand
   \setlength{\textwidth}{14.9cm}
   \setlength{\oddsidemargin}{0.4cm}  % 0cm = 1 inch vom linken Rand
   \setlength{\textheight}{21.6cm}
   \setlength{\topmargin}{+0.5cm}   % 0cm = 1.75 inch vom oberen Rand
%
%
%  % HIER WIRD DIE TITELSEITE GESTALTET
%  Der Titel soll lauten:
%      Bifurkationsanalysen und numerische Simulationen
%      zweier mathematischer Modelle
%      f\"ur selbsterhaltende Systeme.
%
 \title{ \rule[0.0ex]{0.0ex}{0.1ex}   \\
         [-10ex] Bifurkationsanalyse      und    \\
         numerische Simulation    \\
         zweier       mathematischer Modelle \\
         f\"ur      selbsterhaltende Systeme}
%\title{   }
%\author{ {\huge Bifurkationsanalyse      und}    \\
%         {\huge numerische Simulation       }    \\
%         {\huge zweier  mathematischer Modelle} \\
%         {\huge f\"ur   selbsterhaltende Systeme} \\
%       [4ex]  von \\ [4ex] Reinhard Liermann   \\
 \author{ \\ [-4ex] von \\ [4ex] Reinhard Liermann   \\
        [14ex] vorgelegt dem \\
              Fachbereich 1 (Elektrotechnik/Physik) \\
              der Universit\"at Bremen \\
              als Dissertation zur Erlangung des \\
              Grades eines Doktors der Naturwissenschaften \\
              (Dr.\ rer.\ nat.) \\
              [4ex]   }
\date{Bremen  \\    14.\  April  1989}
%
%
                %
                %
\begin{document}
  \pagenumbering{roman}
%
%
%         PAGE I  (TITLEPAGE)
%
% \vspace{-3cm} \rule[0.0ex]{0.1ex}{0.0ex} \newline
  \maketitle
%
%
%         PAGE II
%
  \newpage
      \rule[0.0ex]{0.1ex}{0.0ex}
  \newpage
  \rule[0.0ex]{0.1ex}{0.0ex} \newline
  {\large \underline{Danksagung}    \rule[-2.0ex]{0.0ex}{0.1ex}
  {\em
  \newline
       \begin{minipage}{14cm}
            Ich danke allen, die mir bei der
            Fertigstellung dieser Arbeit
            tat\-kr\"af\-tig
            un\-ter\-st\"utzt
            haben.
            Insbesondere danke ich
            Herrn Prof.\ Dr.\ Helmut Schwegler
            f\"ur die Heran\-f\"uh\-rung an das Thema,
            sowie den Herren
            Dr.\ Ekkehard Mueller und Hagen Voss
            f\"ur  zahlreiche anregende  Diskussionen,
            ohne die diese  Arbeit
            niemals zustande gekommen w\"a\-re.
      \end{minipage}
  }
  }
  \newline
  \rule[0.0ex]{0.0ex}{13.5cm}
  \newline
  {\large
  Gutachter der Dissertation:
       \begin{minipage}[t]{7cm}
               Prof.\ Dr.\ Helmut Schwegler \\
               Prof.\ Dr.\ Peter H.\ Richter
      \end{minipage}
  \newline  \rule[0.0ex]{0.0ex}{2.0ex}
  \newline
  Tag des \"offentlichen Kolloquiums:
                7.\ September 1990
  }
  \newline
  \newpage
%
%
%         PAGE III
%
% \rule[0.0ex]{0.1ex}{0.0ex} \newline
  \begin{center}
     {\large  Bifurkationsanalyse      und
         numerische Simulation \vspace{-0.5ex}   \\
         zweier       mathematischer Modelle  \\
         f\"ur      selbsterhaltende Systeme}
      \\       \rule[0.0ex]{0.0ex}{2ex}  \\
      Reinhard Liermann
      \\       \rule[0.0ex]{0.0ex}{5ex}  \\
      {\large \underline{Abstract}}   \rule[0.0ex]{0.0ex}{-1.0ex}
  \end{center}
% \newline
  Two models of  self-maintaining systems are presented.
  The equation of motion for the surface of the system
    in the first chapter is a function only of the
    surface itself. Moreover, for the sake of simplicity,
    I assume that this is a function
    of the local mean curvature only.
    This equation of motion leads to a
    stationary shape of the system (circle, sphere)
    with a determined size,
    but it also  destabilizes the shape.
    To counteract this fact, a term which behave like a
    surface tension is added.
  \newline
  The model of
    a self-maintaining system in the second chapter is
    based on reaction-diffusion equations.
    Consider a system surrounded by a nutrient solution.
    This system essentially consists of a building-material $A$,
    which I assume  to be homogeneous and incrompressible.
    The nutrient $N$ is diffussing in- and outside the system
    and has a constant          concentration at infinity.
    Inside the system there  are two chemical reactions.
    One from the nutrient
    to the building-material
    and the other from the building-material
    to a decay-material $Z$.
    The decay-material do I not consider,
    cause it should leave the system quickly enough.
    This  process   leads also to
    a stationary shape (circle, sphere) of the system
    with a determined size.
    Again the surface tension stabilizes the shape.
  \newline
  Although the two models of  self-maintaining systems
    look  very different, they have some common properties.
    If the surface tension is strong enough, the
    stationary shape is stable.
    Decreasing the surface tension leads to
    an instability of the surface.
    The dominant mode is     $\ell$=2.
    This means in two dimensions the trigonometrical functions
    $\sin (2\varphi)$ and
    $\cos (2\varphi)$ and
    in three dimensions the spherical harmonics $Y_{2 ,m}$.
  \newline
  But what happens in the unstable region?
    The tools to answer this question are the
    so-called bifurcation analysis (up to second and third order)
    and numerical calculations of the model equations.
  \newline
  For both models there exists
    - in three dimensions -  a certain parameter
    region in which the instability drives the system to a
    division
    (like a cell division).
    But for the model in the first chapter there exist other
    parameter regions, too, where the instability drives the system
    to new stable shapes which look like
    oblate  or prolate spheres
    (like the shape of erythrocytes and  rod-shaped bacteria).
  \newline
  It should be noted, that such models of
    self-maintainig systems, although seeming
    oversimplified compared with real living cells,
    share some essential properties with them.
      %
\end{document}
